\documentclass[cn,fancy,blue,11pt]{elegantbook}


\title{权转移法(Discharging Method)}
\subtitle{学习笔记}

\author{宋宁}
%\translator{宋宁}
\institute{山东理工大学}
\date{\today}


\equote{我们必须知道,我们必将知道 ------ 大卫·希尔伯特}

\logo{timg.jpg}
\cover{cover.jpg}

\usepackage[authoryear]{gbt7714}
\usepackage{makecell}
\usepackage{lipsum}
\usepackage{texnames}
\usepackage{amsmath}
\usepackage{amsfonts}
\usepackage{amssymb}
\usepackage{framed} 

\begin{document}

\newcommand{\setcr}{{\rm CR}}
\newcommand{\nocr}{{\rm cr}}
\newcommand{\mad}{{\rm mad}}
\newenvironment{discharge}{\vskip 10pt\noindent\textbf{\underline{放电规则}}\par}{\par\medskip}


\maketitle

\tableofcontents

\mainmatter
\hypersetup{pageanchor=true}

%%%%%%%%%%%%%%%%%%%%%%%%%%%%%%%%%%%%%%%%%%%%%%%%%%%%%%%%%%%%%%%%%%%%%%%%%%%%%%%%%

\chapter{权转移法简介}

%%%%%%%%%%%%%%%%%%%%%%%%%%%%%%%%%%%%%%%%%%%%%%%%%%%%%%%%%%%%%%%%%%%%%%%%%%%%%%%%%

\section{一个简单的例子}

权转移法,英文名为 Discharging Method,直译即``电荷释放法''.
在图论中,它已经有了超过一百的历史了.
其最著名的应用是证明了四色定理,也就是证明了嵌入在平面上的图的色数至多为$4$.
然而,即使对于大多数图论工作者,这种方法仍然十分神秘.
我们这篇学习指导的目的在于让学习者了解权转移法的使用方法,
从而让更多的人能理解和使用这种方法.

为了解释权转移法,我们看一个简单的例题.
在此之前,我们先来明确一些术语.
如果一个简单平面图的每个面都是$3$-圈,我们称这个简单平面图是一个\textbf{平面三角剖分图}.
如果顶点$v$的度等于$d$,那么我们称$v$为\textbf{$d$-顶点};
如果$v$的度至少$d$(至多$d$),那么我们称$v$为\textbf{$d^+$-顶点}(\textbf{$d^-$-顶点}).
设$P=v_1v_2\dots v_t$是图$G$一条$t$-路.
如果$\deg_G(v_i)=d_i$,其中$i=1,2,\dots,t$,那么我们称路$P$是一个\textbf{$(d_1,d_2,\dots,d_t)$-路};
特别地,如果$\deg_G(v_i)\ge d_i$或$\le d_i$,那么上述数组中的$d_i$可以相应地换成$d_i^+$或$d_i^-$.

\vskip 10pt
\begin{example}\label{emp:ch-in-1}
	如果平面三角剖分图$G$的最小度为$5$,那么$G$包含一条$(5,5)$-边或一条$(5,6)$-边.
\end{example}

\begin{proof}
	首先,我们假设结论是不成立的,而$G$是一个反例.这是权转移法的\textbf{第一步:假设存在反例}.
	
	接下来,我们对$G$的每个顶点$v$赋予$\deg_G(v)-6$个单位的电荷,
	我们一般称之为初始电荷量,记作$ch_0(\cdot)$;在这里,也就是$ch_0(v)=\deg(v)-6$,对每个$v\in V(G)$.
	然后,再给每个面$f$赋予$2\deg(f)-6$个单位的电荷,即$ch_0(f)=2\deg(f)-6$.
	由于$G$是三角剖分图,所以实际上每个面的初始电荷量为零.

	现在我们考虑图$G$上的总电荷量,
	\[\sum_{x\in V(G)\cup F(G)}ch_0(x)=2e(G)-6v(G)+4e(G)-6f(G)=-12<0.\]
	这就是权转移法的\textbf{第二步:赋予初始电荷量}.

	然后是权转移法的\textbf{第三步:转移电荷}.
	我们假象电荷是可以按我们的意志移动的.
	为此我们需要制定一个\textbf{放电规则}(discharging rules).

	我们的放电规则只有一条,那就是:
	每个$5$-顶点从它的邻点拿走$1/5$的电量.

	我们记放电之后每个顶点和面的电荷量为$ch_1(\cdot)$.
	显然,对每个面$f$,$ch_1(f)=ch_0(f)=0$.
	考虑顶点$v$.
	因为$G$是一个反例,所以每个$5$-顶点没有$6^-$-邻点,
	所以如果$\deg(v)=5,6$,那么$ch_1(v)=0$.
	如果$\deg(v)\ge7$,
	由于$G$是三角剖分图且没有$(5,5)$-边,
	所以$v$的$5$-邻点个数不超过$\frac{1}{2}\deg(v)$,
	所以$ch_1(v)\ge\deg(v)-6-\frac{1}{5}\cdot\frac{1}{2}\deg(v)=3/10>0.$

	可见,经过放电后,每个顶点和面的电荷量都是非负,这与总电量小于零矛盾.
\end{proof}

其实上述结论可以加强.
我们称每个面都是3-圈的平面图为\textbf{平面半三角剖分图},
注意平面三角剖分图是简单图,但平面半三角剖分图允许出现平行边
(请大家思考一下为什么有可能出现一个平面图每个面都是3-面同时还有平行边的现象).
上述例题可以加强到平面半三角剖分图.

上述证明还可以简化一下.
因为平面图$G$上必有$e(G)\le 3v(G)-6$.
其等号成立当且仅当$G$是平面半三角剖分图.
所以初始电荷量以及电荷量的计算可以简化.
请大家自行尝试.

\section{构型与不可回避集}

设$G$是一个图.
如果$H$作为$G$的子图满足$G$上的局部性质$\mathcal{Q}$,
那么我们就说$(H,\mathcal{Q})$是$G$的一个\textbf{构型}(configuration),
也称构型$(H,\mathcal{Q})$在$G$上出现了.
如果在上下文语境中,局部性质$\mathcal{Q}$是明确的,
那么我们也可以直接说$H$是$G$的一个构型.

例如,在例\ref{emp:ch-in-1}中,我们可以把边看成子图$K_2$,
那么,$(5,5)$-边和$(5,6)$-边就是两种构型.
如果我们分别记这两种构型为$H_1$和$H_2$,
那么例\ref{emp:ch-in-1}的命题就可以变成:

\begin{framed}
	最小度为$5$的平面三角剖分图含有构型$H_1$或构型$H_2$.
\end{framed}

我们继续来剖析这个命题.
任何命题都有前提和结论.
而任何图论命题的前提都蕴含了一个图族.
例如,如果我们设$\mathcal{G}$是所有最小度为$5$的平面三角剖分图所构成的图族,
并且记$\mathcal{H}=\{H_1,H_2\}$,
那么,在上述命题又可以进一步改写为

\begin{framed}
	$\mathcal{G}$中的每个成员总会含有$\mathcal{H}$中的某个构型.
\end{framed}

一般地,我们设$\mathcal{G}$是一个图族,$\mathcal{H}$是一个由构型组成的集合.
如果$\mathcal{G}$中每个成员都含有至少一个$\mathcal{H}$中的构型,
那么,我们就说构型集$\mathcal{H}$是图族$\mathcal{G}$所无法回避的,
或者说$\mathcal{H}$是$\mathcal{G}$的\textbf{不可回避集}(unavoidable set).

如果我们依然设$\mathcal{G}$所有最小度为$5$的平面三角剖分图所构成的图族,
并且记$\mathcal{H}=\{H_1,H_2\}$,
那么,在上述命题最终可以改写为

\begin{framed}
	$\mathcal{H}$是$\mathcal{G}$的一个不可回避集.
\end{framed}

本质上说,\textbf{权转移法就是搜索某个图族的不可回避集的方法}.
而例\ref{emp:ch-in-1}其实就是Wernicke在1904年提出的第一个使用权转移法解决的图论问题

那么,权转移法是如何搜索不可回避集的呢?
我们需要两个主要工具:初始电荷量和放电规则.

\section{初始电荷量和放电规则}

设$\mathcal{P}$是全体可平面图构成的图族.
权转移法通常在这个图族的某个子族上实施.
设$\mathcal{G}$是$\mathcal{P}$的一个子族.
再设$G\in \mathcal{G}$.
固定图$G$在欧氏平面上的一个平面嵌入.
下面的问题是如何在$G$上设置初始电荷量$ch_0(\cdot)$.
初始电荷量的常见设置方法有三种,即\textbf{顶点充电法}、\textbf{面充电法}和\textbf{均衡充电法}.

顶点充电法的初始电荷量的设置如下:

\[ch_0(x)=\left\{
	\begin{aligned}
		&\deg(x)-6, & x&\in V(G)\\
		&2\deg(x)-6, & x&\in F(G).
	\end{aligned}
\right.\]

使用欧拉公式可以算出,在顶点充电法之下,$G$的初始电荷量的总和为
\[\sum_{x\in V(G)\cup F(G)}ch_0(x)=2e(G)-6v(G)+4e(G)-6f(G)=-12<0.\]

面充电法的初始电荷量的设置如下:

\[ch_0(x)=\left\{
	\begin{aligned}
		&2\deg(x)-6, & x&\in V(G)\\
		&\deg(x)-6, & x&\in F(G).
	\end{aligned}
\right.\]

使用欧拉公式很容易算出,在面充电法之下,$G$的初始电荷量的总和为
\[\sum_{x\in V(G)\cup F(G)}ch_0(x)=4e(G)-6v(G)+2e(G)-6f(G)=-12<0.\]

均衡充电法的初始电荷量的设置如下:

\[ch_0(x)=\deg(x)-4,~~~x\in V(G)\cup F(G).\]

再次使用欧拉公式可得,$G$在均衡充电法下的初始电荷量的总和为
\[\sum_{x\in V(G)\cup F(G)}ch_0(x)=2e(G)-4v(G)+2e(G)-4f(G)=-8<0.\]

在三种常见的充电法里,初始电荷量的总和都是负.
所以如果在某组放电规则下放电之后,出现每个顶点和每个面的电荷量都是非负,
那么一定存在某个矛盾.
也就是说:图$G$原本没有某个性质或结构,
但是我们假定它有,结果导致了矛盾.
实际上,例\ref{emp:ch-in-1}就是这样证明的.

那么如何设计放电规则呢?这就是一件很艺术的事情了.
一般原则上是就近放电,但并不绝对.
事实上,正是由于放电规则设计的不同(以及初始电荷量的不同),
导致了我们能够借助权转移法,找到不同的不可回避集.
而恰恰是这些不同的不可回避集,能在不同的图论问题中起到关键的作用.
下一节我们将给出第二个简单例子,来看一下:变更初始电荷量和放电规则可以给我们带来什么.

\section{第二个简单例子}

我们这一节的任务不是用反证法证明一个现有的问题,而是展示如何使用权转移法搜索$\mathcal{G}$的不可回避集,
从而制造一个定理出来.假定我们讨论的是平面图的某个图族,使用三种常见的初始电荷设置方法,那么
搜索的步骤是这样的:
\begin{framed}
\begin{itemize}
	\item 任取图族$\mathcal{G}$中的一个图$G$;
	\item 对$G$设置初始电荷量(若使用三种常见方法则总电荷量为负数);
	\item 采用一套放电规则对$G$放电;
	\item 找到放电后无法得到非负电荷的所有构型,它们就是$\mathcal{G}$的不可回避集.
\end{itemize}
\end{framed}

下面我们任然以所有最小度为$5$的平面三角剖分图所构成的图族为例,记这个图族为$\mathcal{G}$.
看一下在不一样的初始电荷量和放电规则的前提下,会搜索到什么样的不可回避集.

设$G\in \mathcal{G}$.对$G$设置初始电荷量$ch_0(G)$.令
\[
	ch_0(x)=\left\{
		\begin{aligned}
			& 2\deg(x)-6,	& x\in V(G)\\
			& \deg(x)-6,	& x\in F(G)
		\end{aligned}
	\right.
\]

很明显,面的初始电荷量都是$-3$,而顶点的初始电荷量则都是正数,而且很大.
在这种情况下,常用的思路是:保证顶点非负的前提下,让尽可能多的面非负.
具体来说,一般的设计原则是:
\begin{framed}
	富则平均分担,穷则精打细算.
\end{framed}

比如,在我们现在这个情况中,顶点的度数越大,那么它的初始电荷量就越大.
那么,我们可以形象地说,度数很大的顶点就像富裕的土豪,我们应该把他的财富尽可能平均地分摊出去.
但是,对于度数较小的顶点,我们则要精打细算,从而逼出我们所要找的构型.
这就要求我们首先分析一下,哪些顶点是``富人'',哪些顶点是``穷人''.

设$f$是一个面.
由于$ch_0(f)=-3$,
所以平均来看,$f$所关联的每个顶点至少要给$f$支付$1$个电荷.
但是,如果我们要求某个顶点给它所关联的每个面$1$个电荷,
那么只有这个顶点的度至少为$6$才能支付得起.
所以,面对这个问题时,常规思路会以$6$为分界.
度数大于等于$6$的顶点算``富人'',其电荷平均分配.
度数为$5$的顶点算穷人,需要精打细算.
在这个原则之下,我们制定了如下的放电规则.
\begin{discharge}
	\textbf{规则1.} $6^+$-顶点将它的电荷平均分配给它所关联的所有面.

	\textbf{规则2.} 设$v$是一个$5$-顶点,$f$是$v$所关联的一个面,那么
	\begin{itemize}
		\item 如果$f$是一个$(7^+,5,7^+)$-面,那么$v$把$5/7$个电荷送给$f$;
		\item 如果$f$是一个$(6^-,5,7^+)$-面,那么$v$把$13/14$个电荷送给$f$.
	\end{itemize}
\end{discharge}

事实上,我们不难想到,如果一个$5$-顶点的$6^-$-邻点的个数多于$1$个,
也就是放电规则没有列出的情况,
那么,放电是无论如何也不可能克服这种情况的.
这就是我们要找的不可回避的构型.
换句话说,即:如果平面三角剖分图$G$的最小度为$5$,那么$G$包含一条$(6^-,5,6^-)$-路.

实际上,在我们找到不可回避集的时候,我们也就找到了一条定理,同时找到了它的证明.
如果这条定理的内容足够好的话,我们也就做出来了一篇论文.
请大家自行把它的证明补充完整.

\begin{example}\label{emp:ch-in-2}
	如果平面三角剖分图$G$的最小度为$5$,那么$G$包含一条$(6^-,5,6^-)$-路.
\end{example}

这一节我们解释了权转移法的一个本质,那就是搜索某个图族的不可回避集.
下一节,我们将换一个角度来理解权转移法.

\section{权转移法的另一个本质}

我们前面提到权转移法的本质是搜索某个图族的不可回避集的方法.
如果换一个角度,我们会发现,
权转移法的另一个本质是:
\textbf{一种计数方法}.
为了展现这一点,我们给出例\ref{emp:ch-in-2}的另一个证明.

\vskip 10pt
\begin{proof}
	设$G$是一个最小度为$5$的平面三角剖分图.
	假设$G$不包含$(6^-,5,6^-)$-路.
	给$G$的顶点和面设置初始电荷量$ch_0(\cdot)$如下:
	\[
		ch_0(x)=\left\{
			\begin{aligned}
				&	0,	&	x\in V(G)\\
				&	1,	&	x\in F(G)
			\end{aligned}
		\right.
	\]
	那么$G$的总电荷量就是$f(G)$.下面给出放电规则.

	\begin{discharge}
		\textbf{规则1.} 面$f$向它所关联的每个$6$-顶点转移$1/3$个电荷.

		\textbf{规则2.} 如果面$f$只关联一个$5$-顶点$v$,不关联$6$-顶点,
		那么,$f$向$v$转移$1$个电荷.

		\textbf{规则3.} 如果面$f$恰好关联一个$5$-顶点$v$,一个$6$-顶点,
		那么,$f$向$v$转移$2/3$个电荷.

		\textbf{规则4.} 如果面$f$恰好关联两个$5$-顶点$u,v$,不关联$6$-顶点,
		那么,$f$分别向$u$和$v$转移$1/2$个电荷.
	\end{discharge}
	
	设放电之后的电荷量为$ch_1(\cdot)$.设$v\in V(G)$.
	如果$\deg(v)\ge7$,那么$ch_1(v)=0$.
	如果$\deg(v)=6$,那么$ch_1(v)=2$.
	如果$\deg(v)=5$,因为我们假设$G$没有$(6^-,5,6^-)$-路,
	所以$v$的$6^-$-邻点至多一个,
	那么$ch_1(v)\ge4$.
	设$f\in F(G)$,
	因为条件给定$G$是三角剖分图,所以$\deg(f)=3$.
	如果$f$只关联$6^+$-顶点,那么$ch_1(f)\ge0$.
	如果$f$关联了$5$-顶点$v$,
	因为我们假设$G$不含$(6^-,5,6^-)$-路,
	所以$f$所关联的另外两个顶点,至多一个是$6^-$-顶点.
	如果除了$5$-顶点$v$之外,$f$还关联一个$5$-顶点,
	那么$ch_1(f)=1-1/2-1/2=0$.
	如果除了$5$-顶点$v$之外,$f$还关联了一个$6$-顶点,
	那么$ch_1(f)=1-2/3-1/3=0$.
	总之,如果我们记$d$-顶点的个数为$n_d$,
	那么在放电之后,总的电荷量至少为$4n_5+2n_6$.
	所以\[f(G)\ge 4n_5+2n_6.\]
	但是另一方面,由欧拉公式可得
	\[2=v(G)-e(G)+f(G)=\sum_{d\ge 5}(n_d-\frac{1}{2}dn_d+\frac{1}{3}dn_d)
	=\frac{1}{6}\sum_{d\ge 5}(6-d)n_d.\]
	所以,
	\[12+v(G)=\sum_{d\ge5}(7-d)n_d=2n_5+n_6+\sum_{d\ge8}(7-d)n_d
	\le2n_5+n_6\le \frac{1}{2}f(G),\]
	所以,$f(G)\ge 2v(G)+24$,
	这与$f(G)\le 2v(G)-4$矛盾.
\end{proof}

其实,这个证明与上一节的证明有某种本质上的相通性(但我没想明白).
这个结论其实是Franklin在研究了Wernicke的结论(即例\ref{emp:ch-in-1})后所做的改进.
本节给出的证明是Franklin的原始证明,上一节是我们用现代更常见的形式写出的证明.
然而,Wernicke和Franklin的目的都不仅仅是证明这样两个结构上的结论,
他们研究这两个问题的着眼点都是为了证明四色定理.
这正是权转移法的最著名的应用,
事实上,权转移法的发明最初也就是为了研究四色问题.
那么,染色这种问题是如何用上权转移法的呢?我们需要第三个例子.

\section{第三个简单例子}

这一节我们讨论一下如何利用权转移法研究其他图论问题,比如某种染色.
为此我们这里找了第三个简单例子.
我们记图$G$的平均度为$\deg(G)$,边色数为$\chi'(G)$.

\begin{lemma}{}{lt-e-1}
	如果连通图$G$的平均度小于$3$,那么$G$有$1$-顶点或$(2,5^-)$-边.
\end{lemma}

\begin{proof}
	假设$G$没有$1$-顶点也没有$(2,5^-)$-边.
	对每个$v\in V(G)$设置初始电荷量,令$ch_0(v)=\deg(v)-3$.
	那么$G$的总电荷量为
	\[
		\sum_{v\in V(G)}ch_0(v)=v(G)\cdot \deg(G)-3v(G)<0.
	\]

	放电规则只有一条,那就是每个$2$-顶点从它的每个邻点拿走$1/2$个电荷.
	设放电后$v$的电荷量为$ch_1(v)$.
	如果$\deg(v)=2$,那么$ch_1(v)=-1+2\cdot1/2=0$.
	如果$\deg(v)=3,4,5$,那么$ch_1(v)=ch_0(v)\ge0$,因为不存在$(2,5^-)$-边.
	如果$\deg(v)\ge6$,那么$ch_1(v)=\deg(v)-3-\frac{1}{2}\deg(v)\ge0$.
	因此,放电后的总电荷量全是非负,矛盾.
\end{proof}

设$H$是$G$的子图,令
\[
	w_G(H)=\sum_{v\in V(H)}\deg(v),
\]
我们称$w_H(G)$是$H$在$G$中的重量.
其实,我们到目前为止讨论的所有三个问题四个证明都是关于子图重量的.
例\ref{emp:ch-in-1}是关于边的重量的,例\ref{emp:ch-in-2}是关于$3$-路的重量的,
上面这个引理稍加修改也是关于边的重量的.
这实际是结构图论中的一大类问题,习惯上称为``轻图问题''.
那么这个关于轻图的结论如何用于染色呢?请看下面的定理.

\begin{theorem}{}{e-clr-1}
	如果$\deg(G)<3$并且$6\le\Delta(G)\le k$,那么$G$是$k$-边可染的.
\end{theorem}

\begin{proof}
	假设$G$是极小反例.
	由引理\ref{lem:lt-e-1}可知,
	$G$有$1$-顶点或$(2,5^-)$-边。
	如果$G$有1-顶点$v$,设$v$的邻点为$u$,
	那么$w_G(uv)\le k+1$,
	所以在$G-uv$中,$u$和$v$总共至多关联了$k-1$条边。
	因为$G$是极小反例,
	所以$G-uv$有一个边染色至多用了$k$种颜色。
	但是$u$和$v$总共至多关联了$k-1$条边,
	所以至多用了$k-1$种颜色,剩下一种用来染$uv$,
	那么$G$就有了$k$-边染色,矛盾。
	如果$G$有$(2,5^-)$-边,
	设$xy$是$G$上的一个$2,5^-$-边,
	那么$w_G(xy)\le 7\le k+1$,
	所以仍然可以使用上述论证得到矛盾。
\end{proof}

上面的简单例子很适合说明问题。
简单地说,就是权转移法提供不可回避集,
不可回避集干掉极小反例。
其实,这并不是特殊,这是使用权转移法解决染色问题的一个常见思路。
下一节,我们详细来讨论这个问题。

\section{可归约构型}

我们仍然以染色问题为例,来系统化地解释如何使用权转移法证明问题。
假定我们现在要证明一个图色数的问题,通常是满足某条件的图的色数不能超过某个值。
那么我们需要按如下步骤进行:
\begin{framed}
	\begin{itemize}
		\item 假设结论不成立,取一个极小反例$G$;
		\item 通过条件证明图$G$属于某个``合适的''图族$\mathcal{G}$;
		\item 使用权转移法找到图族$\mathcal{G}$的一个``合适的''不可回避集;
		\item 证明这个不可回避集中的每个构型都不可能在极小反例$G$中出现,从而证明了结论。
	\end{itemize}
\end{framed}

上面所说的,不能在极小反例中出现的构型一般称为\textbf{可归约构型}。
事实上,四色定理就是使用了上面这套方法给出的证明。
要讲清楚这个故事,我们要回到1890年,那时四色定理还是四色猜想。

我们设四色猜想的极小反例为$G$。
1890年,Heawood首先发现$G$属于一个图族,
这个图族正是我们之前反复提到的所有最小度为五的平面三角剖分图的所成的图族,
我们记之为$\mathcal{G}$。
然后,对四色猜想的研究重点转移到了图族$\mathcal{G}$上。
1904年,Wernicke给出图族$\mathcal{G}$的第一个不可回避集,也就是我们的例\ref{emp:ch-in-1},
虽然这个不可回避集只含有两个构型,
而且没有对四色猜想的证明给出直接的帮助,
但是这标志着权转移法正式出现在图论研究的舞台上。
1912年,Birkhoff将前人对四色定理的研究结果和研究方法进行了梳理,
提出了一个系统性解决四色问题的方案,这也就是上面所说的一般步骤的雏形,
事实上,后来真正解决四色问题的不可回避集中就有以Birkhoff命名的Birkhoff构型。
顺着这条路线,Appel和Haken在1976年找到了图族$\mathcal{G}$的一个十分庞大的不可回避集,
这个不可回避集中含有1879个构型,
他们使用汇编语言编写了程序用来验证这些构型是否会在极小反例$G$中出现,
结果表明它正是几代图论学家一百多年的苦苦寻找的那个不可回避集。
但是这个不可回避集实在是大得离谱,
很多图论学者都不承认这是个证明(有人甚至说,这纯粹是个电话号码本)。
虽然Appel和Haken后来精简了这个不可回避集,
但是其基数仍然在一千以上。
直到1997年,Robertson、Sanders、Seymour和Thomas在《组合期刊(B)》上发表了一篇论文,
他们找到了$\mathcal{G}$的一个比较小的不可回避集,
只有633种构型,
并且验证了这些构型都是四色猜想的可归约构型(仍然使用了计算机)。
此时,图论界才正式认为四色定理已经被证明了。
当然,这个证明仍然是不令人满意的,图论界依然等待着更加简洁的、摆脱计算机检验的、可归约的不可回避集。

另一方面,在Wernicke给出了第一个不可回避集之后,
Franklin在1922年推广了Wernicke的结论,他找到了$\mathcal{G}$的一个新的不可回避集,
其中有三个构型,即$(6,5,6)$-路、$(6,5,5)$-路和$(5,5,5)$-路,也就是我们的例\ref{emp:ch-in-2}。
这个工作给分析学家勒贝格以很大启发,
他觉得不要把这么有趣的方法局限于四色问题,
可以考虑其他图族的不可回避集。
为此,他在1940年发表了一篇论文,
列举几个多面体图(即3-连通的平面图)的不可回避集。
随之而来的,就是从1950年代开始,
权转移法开始在结构图论(当然还是以拓扑图为主)的很多问题上广泛开花结果,
这方面的结果之丰富甚至影响到了组合几何学。

后面的几章,我们将从不同方面对权转移法的应用转开讨论。

\begin{exercise}
	如果$G$是一个平均度至少为$2$的简单平面图,那么$G$包含重量至多$15$的边,或包含$2$-交错圈
	(所谓$G$的一个$2$-交错圈是一个$G$的偶圈并交替出现$G$的$2$-顶点)。
\end{exercise}

\begin{proof}
	暂时不说。
\end{proof}

\begin{exercise}
	设$G$是一个平面图。
	如果$g(G)\ge5$并且$\delta(G)\ge2$,
	那么$G$含有$(2,5^-)$-边,或含有$(3,3,3,3,5^-)$-面
	($g(G)$表示$G$的围长,也就是最小圈的长度)。
\end{exercise}

\begin{proof}
	暂时不说。
\end{proof}

%%%%%%%%%%%%%%%%%%%%%%%%%%%%%%%%%%%%%%%%%%%%%%%%%%%%%%%%%%%%%%%%%%%%%%%%%%%%%%%%%

\chapter{轻边}

%%%%%%%%%%%%%%%%%%%%%%%%%%%%%%%%%%%%%%%%%%%%%%%%%%%%%%%%%%%%%%%%%%%%%%%%%%%%%%%%%

\section{Kotzig定理}

设$G$是一个图,$H$是$G$的一个子图.
令
\[w_G(H)=\sum_{x\in V(H)}\deg_G(x).\]
我们称$w_G(H)$是$H$的\textbf{重量}.

设$\mathcal{G}$是一个图族,$H$是一个连通图,使得:每个$G\in\mathcal{G}$都有至少一个子图同构于$H$.
令
\[w(H,\mathcal{G})=\min_{G\in \mathcal{G}}w_G(H).\]
如果$w(H,\mathcal{G})$是有限的,那么我们称$H$是$\mathcal{G}$上的\textbf{轻图},
或者说$\mathcal{G}$有一个轻的$H$.
特别地,如果$H=K_2$,我们说$\mathcal{G}$上有\textbf{轻边}.

对轻图的研究可以追溯到证明四色问题的尝试.
但其正式的起源则是著名的Kotzig定理.
在此,我们首先给出两个概念.

令$\mathcal{M}(\delta,\rho)$表示所有最小点度至少为$\delta$最小面度至少为$\rho$的连通平面图所做成的图族.
从三种常见的放电法,
我们很容易发现,上述$(\delta,\rho)$的取值只能是$(3,3),(3,4),(3,5),(4,3),(5,3)$.
特别地,图族$\mathcal{M}(3,3)$被称为\textbf{正规平面地图}(normal plane map).
如果$G\in \mathcal{M}(3,3)$并且每个面都是$3$-face,那么我们称$G$是\textbf{平面半三角剖分图}.
平面半三角剖分图与平面三角剖分图的区别在于:平面三角剖分图是简单图.

令$\mathcal{P}(\delta,\rho)$表示最小点度至少$\delta$最小面度至少$\rho$的简单的$3$-连通平面图.
之所以强调``3-连通''这个条件,
是因为简单的3-连通平面图等价于3-维欧氏空间上凸多面体的一维骨架(Steinitz定理).

当时,Kotzig先是证明了:$\mathcal{P}(3,3)$上有重量至多$13$的轻边,
$\mathcal{P}(3,4)$上有重量至多$11$的轻边,这两个界都是紧的.
随后,Erd\H{o}s猜想这个结果中的$\mathcal{P}$可以换成$\mathcal{M}$,
这个猜想被Borodin证明了.
我们一般称这个加强版为Kotzig定理.下面我们给出一个使用权转移法的证明.

\begin{theorem}{(KotZig)}{}
	$\mathcal{M}(3,3)$的不可回避集有三个构型:
	\begin{itemize}
		\item $(3,a)$-边,其中$3\le a\le 10$;
		\item $(4,b)$-边,其中$4\le b\le 7$;
		\item $(5,c)$-边,其中$5\le c\le 6$.
	\end{itemize}
	其中,$10,7,6$三个界都是紧的.
\end{theorem}

\begin{proof}
	假设结论不成立.设$G$是所有顶点数$n$的反例中边数最多的(即极大反例).
	那么,$G$的边必是如下构型之一:
	\begin{itemize}
		\item $(3,11^+)$-边;
		\item $(4,8^+)$-边;
		\item $(5,7^+)$-边;
		\item $(6^+,6^+)$-边.
	\end{itemize}
	而且$G$的每个面都是$3$-面.
	这是因为:假设$G$有一个面$f$不是$3$-面.
	如果$f$关联一个$3$-顶点,那么这个$3$-顶点在$f$的边界上关联了两个$11^+$-顶点,
	连接这两个$11^+$-顶点,仍然会得到$n$个顶点的反例,但是边数多于$G$矛盾.
	类似地,如果$f$关联$4$-、$5$-、$6^+$-顶点,我们总可以给$G$加边.

	对每个$v\in V(G)$设置初始电荷量$ch_0(v)=\deg(v)-6$.由于$G$的每个面都是$3$-面,所以
	\[\sum_{v\in V(G)}ch_0(v)=2e(G)-6v(G)=-12<0.\]

	\begin{discharge}
		\textbf{规则1.} 如果$uv\in E(G)$使得$\deg(u)\ge7$并且$\deg(v)\le 5$,
		那么$u$把$\frac{6-\deg(v)}{\deg(v)}$个电荷送给$v$.
	\end{discharge}
\end{proof}






\bibliographystyle{plain}
\bibliography{reference}

\appendix

\chapter{A}



\end{document}
